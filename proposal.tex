% !TEX root = main.tex
\section{Proposal} % (fold)
\label{sec:proposal}

Our goal is to build a system that helps developers understand how their program and the environment effect the operation run time of their device.
The purpose of this system is to enable developers to make informed decisions to increase the longevity of their devices run time.
By creating a tool to provide this information we can later feed this information back to a system like Flicker \cite{flicker} to facilitate dynamic run time changes based on energy harvesting conditions.
Feeding this information to a dynamic system like Flicker is the end goal, however it is too ambitious to complete in a semester project.
To accomplish this goal we need to quantify the energy harvesting rate as well as the energy usage of their application.

To quantify energy harvesting rates we will build a batteryless device that uses solar energy harvesting systems.
Our goal is to use the system in as many different harvesting environment systems as possible.
Time permitting we may investigate vibration, RF, and/or thermal energies.
We will use a current sense chip on a batteryless shield we have designed for the MSP430FR5969 TI Launchpad to measure power level.

Given the time constraint, we will measure the energy usage of applications only using a static code analysis tool written with ANTLR \cite{antlr}.
In the future we would like to implement a run time analysis component.
We will write two applications of varying complexity and determine their energy use with static code analysis.
The first application will be the simplest - it will only take a sensor reading and save it in FRAM.
The next application will add the ability to send data via the radio.
Depending on time constraints we may add additional applications that either run a simple computation or send data at higher frequency rates.

The system will consist of two launchpads one with a batteryless shield powered by a solar cell and another powered via USB power to a laptop computer.
The applications mentioned earlier will be annotated with tags to identify tasks and flows as well as data expiration.
The Java ANTLR application will strip this information out and replace the tags with relevant C code for communicating status to the usb-powered launchpad.
From there the usb powered launchpad will communicate with a Java program on a laptop to save and/or plot the data received from the solar launchpad.
