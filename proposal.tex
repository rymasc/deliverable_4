% !TEX root = main.tex
\section{Proposal} % (fold)
\label{sec:proposal}

Our goal is to build a system that helps developers understand how their program and the environment effect the operation run time of their device.
The purpose of this system is to enable developers to make informed decisions to increase the longevity of their devices run time.
By creating a tool to provide this information we can later feed this information back to a system like Flicker \cite{flicker} to facilitate dynamic run time changes based on energy harvesting conditions.
Feeding this information to a dynamic system like Flicker is the end goal, however it is too ambitious to complete in a semester project.
To accomplish this goal we need to quantify the energy harvesting rate as well as the energy usage of their application.

To quantify energy harvesting rates we will build a device that measures energy harvesting systems for solar, vibration, RF, and thermal energies.
Our goal is to measure as many harvesting systems as possible, though for time constraints we will start with solar and vibration.
The system will determine the amount harvested as well as if it will support applications.
In addition, it will be able to recommend which particular form of energy harvesting would be best based on the environment.

Given the time constraint, we will measure the energy usage of applications only using a static code analysis tool written with ANTLR \cite{antlr}.
In the future we would like to implement a run time analysis component.
We will write two to three applications of varying complexity and determine their energy use with static code analysis.
The first application will be the simplest.
It will only take a sensor reading and save it in FRAM.
The next application will add the ability to send data via the radio.
Depending on time constraints we may add additional applications that either run a simple computation or send data at higher frequency rates.
